\documentclass[a4paper,11pt,french]{article}

\usepackage[utf8]{inputenc}             
\usepackage[T1]{fontenc}
\usepackage{microtype}
\usepackage{lmodern}
\usepackage[margin=3cm]{geometry}
\usepackage{hyperref}
\usepackage{pgfplots}
\usepackage{listings}

\title{Architecture des Systèmes\\Compte-rendu Ordonnanceurs Nanvix}
\author{Antonia IVANOVA\\Rémi GAULMIN}

\begin{document}
\lstset{
  belowcaptionskip=1\baselineskip,
  breaklines=true,
  frame=L,
  xleftmargin=\parindent,
  language=C,
  showstringspaces=false,
  basicstyle=\footnotesize\ttfamily,
  keywordstyle=\bfseries\color{green!40!black},
  commentstyle=\itshape\color{purple!40!black},
  identifierstyle=\color{blue},
  stringstyle=\color{orange},
}
\maketitle

\tableofcontents

\section{Ordonnancement par priorités}

\subsection{Choix dans la politique d'ordonnancement}

Notre ordonnanceur à priorités gère deux niveaux de priorités pour un processus :

\begin{itemize}
	\item La priorité système, avec le champ \texttt{priority};
	\item La priorité utilisateur, avec le champ \texttt{nice}.
\end{itemize}

Nous avons fait le choix de donner la priorité à la priorité système plutôt qu'à la priorité utilisateur.
Les deux priorités fonctionnent de la même manière : plus le champ priorité est bas, plus le processus est prioritaire.
\\

Nous avons également pensé au problème que représente la famine : dans la politique d'ordonnancement par priorités, un processus prioritaire peut ne jamais donner la main aux autres processus.
C'est pourquoi nous avons modifié légèrement la politique:

On fait la soustraction de la priorité système par le temps d'attente de chaque processus : ainsi, le processus qui a attendu le plus en fonction de sa priorité système aura la main sur le processeur.
À priorité système égale et temps d'attente égale, on considère la priorité utilisateur.
À priorité utilisateur égale, on applique la politique d'ordonnancement de type tourniquet.
\\

\subsection{Les problèmes rencontrés}

Nous avons passé beaucoup de temps sur cet ordonnanceur, car nous avions du mal à comprendre pourquoi nous obtenions le message \texttt{core dumped} sur notre première version d'implémentation.
Ce message indiquait qu'il y avait des surcharges mémoires au niveau utilisateur.
Cela était dû au fait que les compteurs d'attente des processus s'incrémentaient après l'analyse des priorités.


\section{Files d'attentes multiples}

\subsection{Choix dans la politique d'ordonnancement}

Notre ordonnanceur à files d'attentes multiples gère des catégories de priorités.
Les catégories de priorités sont, dans l'ordre décroissant de priorité : 1, 2, 4, 16, 32 et 64.
\\

Afin de pouvoir se souvenir de la catégorie pour chaque processus, nous avons rajouté un champ à la structure \texttt{process}, nommé \texttt{mlqclass}, et initialisé dans les fonctions \texttt{sys\_fork} et \texttt{pm\_init}.

Le principe est de multiplier par $2^{mlqclass}$ le quantum alloué à un processus.
Ainsi, il nous suffit de trouver le processus avec la catégorie de priorité la plus faible, et s'il y en a plusieurs, appliquer la politique du tourniquet entre ces processus de même catégorie.



\section{Ordonnancement par tirage au sort}

\subsection{Choix dans la politique d'ordonnancement}

Pour cette partie, nous avons fait le choix d'implémenter la politique de la loterie selon l'algorithme suivant : 
\\

\begin{enumerate}
	\item Chaque processus dans la file des prêts se présente et se voit attribuer un certain nombre de tickets de loterie (ce nombre est décidé de façon arbitraire). Le processus est ensuite rangé dans une liste.
	\item Un nombre est aléatoirement tiré au sort entre 0 et le nombre total de tickets de loterie distribués (on appellera $x$ ce nombre).
	\item Pour retrouver le processus qui a le ticket $x$, nous parcourons la liste où nous avons stocké nos processus dans l'ordre dans lequel ils sont arrivés.
		\begin{itemize}
			\item Si nous sommes au $i$-ème processus de notre liste qui possède k tickets, nous vérifions si $x \le [$somme du nombre de tickets des processus aux indices entre 0 et $i-1] + k$
				\begin{itemize}
					\item Si oui, nous sortons de la boucle
					\item Sinon nous réitérons avec le $i+1$-ème élément
				\end{itemize}
		\end{itemize}
	\item  Une fois que nous avons retrouvé le processus tiré au sort, on appelle la fonction \texttt{switch\_to()} avec comme argument ce processus. 
\end{enumerate}

\subsection{Les problèmes rencontrés}

Même si nous essayons de faire fonctionner notre ordonnanceur (i.e. qu'il passe les tests correctement), nous avons un message \texttt{core dumped} qui persiste. 
Nous n'arrivons pas à trouver la raison de ce message. 

Nous avons exploré plusieurs pistes afin d'essayer de résoudre ce problème, notamment de revoir notre mécanisme de distribution de tickets. 
En effet, comme ces derniers sont distribués en un nombre que nous avons fixés de façon arbitraire, nous nous sommes demandés si le problème ne venait pas de là. 
Il semblerait que non puisque chaque processus a une chance non nulle d'obtenir le processus, peu importe donc le nombre non nul de tickets qu'on donne à un processus.



\section{Les tests}

Nous avons rajouté un test (\texttt{sched\_test4()}) qui prend deux processus (le père et le fils) avec des champs nice différents. Lorque le processus finit son exécution, il affiche un message pour indiquer qu'il a fini.

Il est à noter que nous avons rajouté ce test dans toutes les versions de notre ordonnanceur.

\end{document}
